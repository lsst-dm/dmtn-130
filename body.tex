\section{Introduction}
Tony Tyson has been contacted by Bill Gates with the idea to have some part of the DM technical stack named
after Jim Gray.  Gray while at IBM was one of the founders of relational databases, at Microsoft research he gave us
Skyserver and helped make CasJobs work.
As DM project manager and someone who worked with Jim on SDSS I think this is an excellent idea.  There was mention of some donation for this too but that is really icing - the fact the Gates would like to honor Jim Gray in some way through LSST is fitting and good. This document gives some ideas on the topic. We need a list of potential candidates for Tony to discuss with Gates.

We are totally unclear on how much the agencies may care about any of these.
Something to discuss here may be hosting "the archive" on Azure for free or highly discounted.

\section{Things to name}

We could look at our oft maligned product tree in DM for this. Under software we have :
\begin{itemize}
\item Batch Production
\item Database Back Bone
\item LSST Science Platform
\item Prompt Processing
\item Science Pipelines
\item Quality Control
\item Supporting - including Qserv, Butler, task, ADQL and imgServ.
\end{itemize}

Of the above not all are useful - QC, Batch, DBB, are not very visible. ADQL seems small though relevant.

\subsection {The Archive}
Not mentioned in the product list is the archive, this is somewhat related to \secref{sec:butler}. The archive is
the collection od images and catalogs from LSST. It will always exist even after LSST. It is accessed through the Science Platform (\secref{sec:sciplat}) and Qserv (\secref{sec:qserv}) is part of that. Any of the components may change but the
archive would always be there.

This is probably the least contention item in DM for this purpose. It is also highly visible and NSF may care if we start using the term Jim Gray LSST Archive or such.

\subsection {The Data Facility}
In operations the NCSA end of things will be labeled the LSST Data Facility  and might provide an opportunity. It is a physical location and so could have a plaque. Physical locations and plaques will almost certainly draw the interest/involvement of NSF which may complicate things.

\subsection{Qserv } \label{sec:qserv}
In terms of appropriateness, Gray being one of the SQL founders, a database would be very appropriate, a relational
system like Qserv would be the top thing. \footnote{Jim's Turing Prize citation was : "For seminal contributions to database and transaction processing research and technical leadership in system implementation"} One worry may be the lifetime of such a product. According to Tony there is no
indication of a need to be Microsoft1t based. One wonders though if a good path forward for Qserv might indeed be a collaboration with Microsoft to make a Qserv on Sqlserver, something which would make MyDB easier and could actually lead to a long term product like the Jim Gray Petascale DB, or perhaps Grayscale DB.

We are unsure of the status of Graywulf from JHU which was a DB system named in honor of Jim Gray.
Qserv is post Jim Gray era -  and they are a little hesitant on this option.

\subsection{Butler - LSST Database} \label{sec:butler}
We do not really think of the Data Release as a database but one could consider the data system underlying the science platform as a database and name it. This would give longevity as it will always exist even if the technology changes.
The front end manifestation of this is the Butler, which contains the relational registry of all image metadata.
I imagine it might amuse Jim if in our code we had jim\_gray.get(\ldots) and jim\_gray.put(\ldots), some might find it disrespectful.
 This would require some code change but if there is a large donation it may be doable.

\subsection{Prompt Processing}
There are two things here which one could potentially name: the alert stream and the prompt products database.
Both will be long lived in the project. The alert stream is of course one of th highest profile parts of LSST.
On May imagine most people would still call it the alert stream but it it could be branded and referred to officially as
the Jim Gray Alert Stream or such.

\subsection{Science Platform} \label{sec:sciplat}
 The science platform will definitely be long lived - even if the technology changes the name will stick, So one could consider this an viable option. One may in this case at least want to consider an Azure deployment, they were at least considering supporting K8s (check).
Jim was not a big proponent of open software and this is a large open software project (true for all DM) - so there is some reluctance the use the science platform for this opportunity.


\subsection{imgServ}
The image service could be an option. It will be long lived, given Jim's association with Skyserver and Terraserver this
would be appropriate. It is an underlying service and may not be very visible though.

